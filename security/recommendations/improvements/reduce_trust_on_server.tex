\subsection{Reduce Trust on Server}

\paragraph{Problem:}
Users have to trust the server to deliver the correct client code.

\paragraph{Consequences:}
The threat model is reduced to a honest-but-curious server.
Without the limitation of trusting correct client code delivery, CryptPad could defend against an attacker with more active capabilities.

\paragraph{Suggestions:}
There are multiple approaches:
\begin{enumerate}
  \item Providing an API so that a client can be distributed over an independent channel (App-store, GitHub, Browser extensions, ...).
  \item Trust on First Use (TOFU): users are warned when their client code changes and are asked whether they want to accept the update or not.
  \item Sign the client code and allow users to verify the signature.
\end{enumerate}

\paragraph{Drawbacks:}
\begin{enumerate}
  \item It gets more complicated to use CryptPad, when users first need to download a client. Hence, the webclient should still be accessible.
  \item Currently, all users have the same client version. If this is no more the case, there might be incompatible features, or even \enquote{client fights}, i.e., clients are resolving patches differently and thus always trying to push their version of the document.
  \item Some measures such as signed code may only be effective if the users verify them. As experience in the context of HTTPS verification with extended validation have shown~\cite{Keizer2019}, this might not be the case.
  \item The code should not only be verified for the flagship instance, but also for custom instances. However, they might want to customize the client code.
\end{enumerate}
