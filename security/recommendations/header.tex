% \usepackage{bytefield}
\usepackage{enumitem}
\usepackage{changelog}
\usepackage{hyperref}
\usepackage[T1]{fontenc}
\usepackage{siunitx}
% \sisetup{text-family-to-math = true, text-series-to-math = true, group-separator = {,}}
\usepackage[ main=english,]{babel}
%% These additional packages are used within the document:
\usepackage{booktabs}  % Tables
\usepackage{multirow}
\usepackage{tabularx}
\usepackage{tikz}      % Diagrams
\usetikzlibrary{arrows, calc, shapes, backgrounds, shadows, positioning}
\usetikzlibrary{fit}
\usetikzlibrary{patterns}

\tikzstyle{int}=[draw, fill=blue!20, minimum size=2em]
\tikzstyle{init} = [pin edge={to-,thin,black}]
\tikzstyle{bits} = [font=\footnotesize]
\usepackage{pgf}
\usepackage{pgf-umlsd}
\usepackage[utf8]{inputenc}
\usepackage{lipsum}
\usepackage{xspace}
\usepackage{calc}


\setlength{\marginparwidth}{1.5cm}
\usepackage{todonotes}
% \usepackage{natbib}
\usepackage{csquotes}
\usepackage[nolist]{acronym}

\usepackage[backend=biber, style=ieee, sortcites, maxcitenames=1,mincitenames=1]{biblatex}
\AtEveryBibitem{
   \clearfield{month}
   \clearfield{series}
%    \clearfield{venue}
   \clearname{editor}
   \clearlist{publisher}
   \clearlist{location} % alias to field 'address'
   \clearfield{eprint}
   \clearfield{doi}
%    % \clearfield{url}
   % \clearfield{venue}
   \clearfield{issn}
   \clearfield{isbn}
   \clearfield{note}
%    \clearfield{urldate}
%    \clearfield{urlday}
%    \clearfield{urlmonth}
%    \clearfield{urlyear}
   \clearfield{eventdate}
   \clearfield{pages}
   \clearlist{language}
   % \clearfield{booktitle}
   % \clearfield{journaltitle}
   \clearfield{volume}
    \ifentrytype{article}{
      \clearfield{number}
      \clearfield{url}
      \clearfield{urlday}
      \clearfield{urlmonth}
      \clearfield{urlyear}
  }{}
    \ifentrytype{inproceedings}{
      \clearfield{number}
      \clearfield{url}
      \clearfield{urlday}
      \clearfield{urlmonth}
      \clearfield{urlyear}
  }{}
}


\usepackage{stackrel}
\usepackage{amsmath, amssymb}
\usepackage{nicefrac}

\usepackage{float}
\floatstyle{plain}
\newfloat{lstfloat}{tpbh}{lop}
\floatname{lstfloat}{Listing}
\def\lstfloatautorefname{Listing}

\usepackage{xurl}       % `\url`s
\usepackage{listings}  % Code listings

\usepackage{tablefootnote}
\usepackage{pifont}


\lstset{%
  basicstyle=\small\ttfamily,
  breakatwhitespace=false,
  breaklines=true,
  commentstyle=\color{green!60!black},
  extendedchars=true,
  keywordstyle=\color{blue},
  showspaces=false,
  showstringspaces=false,
  showtabs=false,
  stringstyle=\color{violet},
  numberstyle=\tiny\color{gray},
  numbers=left,
  numbersep=5pt,
  frame=single,
  tabsize=2,
}
% Fix space after caption,
% see https://tex.stackexchange.com/questions/248804/ieee-latex-listing
\makeatletter
\def\lst@makecapton{%
  \def\@captype{table}%
  \@makecaption
}
\makeatother

% \usepackage{fontawesome5}

% Crypto stuff
% \usepackage{algorithm}
\usepackage[ n, advantage, operators, sets, adversary, landau, probability,
notions, logic, ff, mm, primitives, events, complexity, asymptotics, keys,
% oracles
]{cryptocode}
% \usepackage{algpseudocode}
\usepackage{algorithmicx}

\usepackage[capitalise]{cleveref}
\crefname{enumi}{}{}

\newcommand{\authnote}[2]{{\bf \textcolor{blue}{#1}: \em \textcolor{red}{#2}}}

\newcommand*\justify{%
  \fontdimen2\font=0.4em% interword space
  \fontdimen3\font=0.2em% interword stretch
  \fontdimen4\font=0.1em% interword shrink
  \fontdimen7\font=0.1em% extra space
  \hyphenchar\font=`\-% allowing hyphenation
}

\renewcommand{\texttt}[1]{%
  \begingroup
  \ttfamily
  \begingroup\lccode`~=`/\lowercase{\endgroup\def~}{/\discretionary{}{}{}}%
  \begingroup\lccode`~=`[\lowercase{\endgroup\def~}{[\discretionary{}{}{}}%
  \begingroup\lccode`~=`.\lowercase{\endgroup\def~}{.\discretionary{}{}{}}%
  \catcode`/=\active\catcode`[=\active\catcode`.=\active
  \justify\scantokens{#1\noexpand}%
  \endgroup
}

\setlist[enumerate]{
  itemsep=1.5pt,
  labelwidth=8pt,
  leftmargin=10pt,
  labelsep=3pt,
  listparindent=\parindent,
  parsep=1.5pt,
  % parskip=1.5pt,
  topsep=1.5pt,
}


% Definitions
\newcommand{\myAlgofont}[1]{\ifmmode\mathsf{{#1}}\else\textsf{{#1}}\fi\xspace}
\newcommand{\myHash}[1]{\ensuremath{\myAlgofont{H}({#1})}}
\newcommand{\myKGen}[2][]{\ensuremath{\myAlgofont{KGen}\ifthenelse{\equal{#1}{}}{}{#1}({#2})}}
\newcommand{\mypwd}{\myAlgofont{pwd}}
\newcommand{\mySalt}{\myAlgofont{salt}}
\newcommand{\myUsername}{\myAlgofont{user name}}
\newcommand{\myseed}{\myAlgofont{seed}}
\newcommand{\mychanID}{\myAlgofont{chanID}}
\newcommand{\mySuperseed}{\myAlgofont{superseed}}
\newcommand{\myViewKeyStr}{\myAlgofont{viewKeyStr}}
\newcommand{\myFileKeyStr}{\myAlgofont{fileKeyStr}}
\newcommand{\myEditKeyStr}{\myAlgofont{editKeyStr}}
\newcommand{\myPK}{\myAlgofont{PK}}
\newcommand{\mySK}{\myAlgofont{SK}}
\newcommand{\myKeyPair}[1]{\ensuremath{(\myPK{#1}, \mySK{#1})}\xspace}
\newcommand{\mySymKey}{\myAlgofont{K}}
\newcommand{\mySymEnc}[2]{\ensuremath{\myAlgofont{SymEnc}({#1}, {#2})}}
\newcommand{\myASymEnc}[1]{\ensuremath{\myAlgofont{ASymEnc}({#1})}}
\newcommand{\myCtxt}{\ensuremath{c}}
\newcommand{\mySig}{\ensuremath{sig}}
\newcommand{\myKDF}[2]{\ensuremath{\myAlgofont{KDF}({#1}, {#2})}}
\newcommand{\mySign}[2]{\myAlgofont{Sign}({#1}, {#2})}

