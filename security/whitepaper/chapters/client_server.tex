\section{Client-Server Communication}
\label{sec:client_server}
In this section, we show how we use the NetFlux protocol~\cite{DeLisle2016} to enable communication between web clients and the server.
We discuss the usage of \ac{TLS} that helps to defend against \ac{MITM} and impersonation.

Each CryptPad instance makes use of a central server:
while not being able to read any content, the server still acts as an intermediary between different clients.
This is not only to forward the messages, but also to store them and to guarantee a highly available persistence layer that would not be possible in a purely peer-to-peer solution.


Communication between the client and the server is mostly done using a Websocket connection based on the NetFlux protocol.
Exceptions are static files (images, videos, PDF, etc.) that are stored in an encrypted format and are retrieved by users with \acp{XHR}.

With the NetFlux protocol, users  can join \textit{channels} and send messages to them.
All users subscribed to a channel will then receive the messages.

Each document is represented on the server by a channel with a unique 32-character hexadecimal identifier called \mychanID.
The \mychanID associated with a document is derived from its URL, so that users knowing the URL can subscribe to the channel.

While a channel provides a way to communicate messages or changes to a document, channels do not provide confidentiality or authenticity.
We therefore use \ac{TLS} to encrypt the messages between the server and the client.
This allows to authenticate the server and therefore rules out \ac{MITM} attacks.
It further restricts any network adversary from accessing the encrypted content.

However, \ac{TLS} on its own is not enough as the encryption is only between the server and the client, but not between two clients.
\ac{TLS} can moreover not prevent clients from subscribing to channels they should not have access to.
To prevent against this, we make use of dedicated encryption schemes hand tailored to CryptPad.
We present these schemes in the \cref{sec:pads,sec:cryptdrive,sec:mailbox,sec:teams}.
