\section{Teams}
\label{sec:teams}

In this section, we present CryptPad's team encryption which is similar to the general messaging encryption, but suitable for the use case where we want to have fine-grained control over reading and writing access.
We first show in \cref{sec:team_key_derivation} how we derive the team's keys and how we use them for message encryption and the control of the team's drive.
Next, we present in \cref{sec:roles_and_permissions} the different roles and permissions in team and how we enforce them.

\subsection{Key derivation}
\label{sec:team_key_derivation}
We want that all users of a team can read messages, but not necessarily all of them to be able to write to the mailbox.
We therefore generate not only an encryption key pair, but also a signing key pair which allows proving writing capabilities.

\begin{figure}[t!]
  \centering
  %! TeX root = ../main.tex

% Set minimum height of boxes (especially 'K' and 'y').
\newlength{\myminheightteam}
\settoheight{\myminheightteam}{$\mySK_{T,S}$}

\begin{tikzpicture}[minimum height=\myminheightteam,node distance=1.2cm,auto,>=latex']
  \def \ydist{1.3cm}
  \node (seed) [init] {$\myseed$};

  \node (mid) [below of=seed, coordinate, node distance=\ydist] {};
  \node (left) [left of = mid] {};
  \node [int] (hash) [left of=left] {$\myHash{\cdot}$};
  \node [int] (kgens) [right of=mid] {$\myKGen[_S]{\cdot}$};

  \node (mid2) [below of=mid, coordinate, node distance=\ydist] {};
  \node (left2) [left of = mid2, coordinate] {};
  \node (belowhash) [below of = hash, coordinate, node distance = \ydist] {};
  \node [int] (kgene) [below of =hash, node distance = \ydist] {$\myKGen[_E]{\cdot}$};

  \node (SKE) [below of = kgene, node distance = \ydist] {$\mySK_T$};
  \node (PKE) [right of = SKE] {$\myPK_T$};
  \node (chanID) [right of = PKE] {\mychanID\vphantom{$\mySK{_{T,S}}$}};
  \node (SKS) [right of = chanID] {$\mySK_{T,S}$\vphantom{$\mySK_{T,S}$}};
  \node (PKS) [right of = SKS] {$\myPK_{T,S}$\vphantom{$\mySK_{T,S}$}};

  \def \offset{1mm}

  \draw[->] ($(seed.south) - (\offset, 0)$)  |- ($(seed)!0.5!(hash)$) node[above, bits] {0..31} -| (hash);
  \draw[->] ($(seed.south) + (\offset, 0)$) |- ($(seed)!0.5!(kgens)$) node[above, xshift=1mm, bits] {32..63} -| (kgens);


  \draw[->] (hash) -- node [left, bits] {0..31} (kgene);
  \draw[->] ($(hash.south) + (2*\offset, 0)$) |- ($(hash)!0.25!(chanID)$) node[above, xshift=5mm, bits] {32..47} -| (chanID);

  \draw[->] (kgene) -- (SKE);
  \draw[->] ($(kgene.south) + (2*\offset, 0)$) |- ($(kgene)!0.5!(PKE)$) -| (PKE);

  \draw[->] (kgens) -- (SKS);
  \draw[->] ($(kgens.south) + (2*\offset, 0)$) |- ($(kgens)!0.5!(PKS)$) -| (PKS);
\end{tikzpicture}

  \caption{Derivation of team keys}
  \label{fig:team_keys}
\end{figure}

The key generation for a team is depicted in \cref{fig:team_keys}.
We sample 18 Bytes uniformly at random for \myseed and split it into two halves.
We hash the first half and build from this hash the \mychanID and an encryption/decryption key pair $\myKeyPair{_T}$.
The second half forms the input to $\myKGen[_S]{\cdot}$ which generates a signing key pair $\myKeyPair{_{T,S}}$.

To send a message to a team, the user encrypts it with the same mechanism as described in \cref{sec:mailbox}.
However, the user must set the team's $\myPK_T$ as the public key of the receiver and use the team's $\mySK_{T,S}$ as the signing key.

The public signing key is further used to control a team's drive which works the same way as one of a single user.
For example, we can use the signing key to pin a document, i.e., to indicate to the server that a specific document is owned.
In the perspective of the server, the public key of this pin is, however, undistinguishable from the one of a single user.
Based on an offline view of the database (i.e., after a seizure of law enforcement), the server can thus neither know which entities are teams, nor can the server match users to teams.

\subsection{Roles and Permissions}
\label{sec:roles_and_permissions}

\begin{table}[t!]
  \centering
  \caption{Different roles and their permissions}
  \label{tab:roles_and_permissions}
\begin{tabular}{@{}lcccc@{}}
\toprule
Role    & View & Edit & \parbox[c]{1.4cm}{Manage\\  Members} &  \parbox[c]{1.4cm}{Manage\\ Team} \\ \midrule
Viewer & \checkmark &      &  &\\
Member & \checkmark & \checkmark & &\\
Admin  & \checkmark & \checkmark & \checkmark &\\
Owner  & \checkmark & \checkmark & \checkmark & \checkmark \\ \bottomrule
\end{tabular}
\end{table}

As shown in \cref{tab:roles_and_permissions}, there are four different roles in a team: viewers, members, admins, and owners.
The permissions are the following:
\begin{description}[itemsep=0pt]
  \item[View:]  read-only access to folders and documents.
  \item[Edit:]  create, modify, and delete folders and documents.
  \item[Manage users:]  invite and revoke users, change user roles up to admin.
  \item[Manage team:]  change team name and avatar, add or remove owners, change team subscription, delete team.
\end{description}
The number of permissions is strictly increasing, i.e., a role inherits all access rights from a weaker one, but has one additional permission.

All users keep a local copy of their teams including their rosters.
To manage members and the team itself, admins (respectively owners) send the corresponding control message to all users of the team.
They then check the authenticity of the message and whether the sender has sufficient rights to perform the desired action.
If this is the case, then they update their local view of the team.

In order to, e.g., add a new document to the team's drive, the user sends the viewing keys to the viewers, and the editing keys to the teams' members, admins, and owners.
