\section{Messaging}
\label{sec:mailbox}

In this section we present CryptPad's own end-to-end encrypted messaging system that allows users to exchange arbitrary messages and metadata with other users.
The messaging system is further used for, e.g., instance support, team messaging (c.f. \cref{sec:teams}) and forms.
An important property of the used encryption scheme is anonymity: A user eavesdropping on another user's mailbox can not infer the sender of the message.

The basic block of the message encryption is to use public-key authenticated encryption with $\myAlgofont{Nacl.box}(\cdot)$ which internally derives a shared key between the receiver's and the sender's keys.

We build our encryption $\myASymEnc{\cdot}$ on this by encrypting a plaintext under the sender's public/private keys $\myKeyPair{_A}$ and the receiver's public key $\myPK_B$  as follows:
\begin{align*}
&\myASymEnc{\myPK_A, \mySK_A, \myPK_B, m}\\[-2.5ex]
&\rule{\widthof{$c \gets \myAlgofont{Nacl.box}(m, N, \myPK_B, \mySK_A)$}}{0.4pt}\\ % Draw rule with width of 'c <- ...'
&N \sample \bin^{192}\\
&c \gets \myAlgofont{Nacl.box}(m, N, \myPK_B, \mySK_A)\\
&\pcreturn N || c || \myPK_A
\end{align*}
The 24-bytes-sized nonce $N$ is sampled uniformly at random and is prepended to the authenticated ciphertext $c$.
The sender's public key $\myPK_A$ is further appended to indicate the sender's identity to the receiver.

The receiver can then split the received message into $N$, $c$, and $\myPK_A$ to decrypt it using $\myAlgofont{Nacl.box.open}(c, N, \myPK_A, \mySK_B)$.
In case that the sender wants to decrypt this message, the server must use $(\myPK_B, \mySK_A)$ instead of $(\myPK_A,\mySK_B$), since the sender does generally not have access to $\mySK_B$.

However, we do not directly encrypt messages using $\myASymEnc{\cdot}$ as this would leak the sender of the message.
Instead, we apply a second layer of encryption using \textit{ephemeral} keys which are freshly generated for each message and thus are not linkable to the sender.
Lastly, we may additionally sign the double encrypted messages using a signing key to prove write access.

An exemplary use case for this is the signing of a form answer: the users will receive it from the author to prove that they actually are allowed to send messages answering the form.
For other cases the signing key may be obtained through the accounts' profile pages, or as a point of contact in documents.

\begin{figure}[t!]
  \centering
  %! TEX root = ../main.tex
\tikzstyle{int}=[draw, fill=blue!20, minimum size=2em]
\tikzstyle{init} = [pin edge={to-,thin,black}]
\tikzstyle{enc} = [minimum height=1.5cm,text width=2.5em,align=center,fill=blue!20,draw]
\tikzstyle{ifelse} = [font=\footnotesize]

\begin{tikzpicture}[node distance=1.3cm,auto,>=latex']
  \def \ydist{1.5cm}
  \def \encdist{0.5cm}
  \node (m) {$m$};

  \node (enc1) [enc, below of=m, node distance=\ydist] {\myAlgofont{Enc}};
  \node (left) [left of=enc1, coordinate] {};
  \node (PKA) [ left of=left] {$\myPK_A$};
  \node (SKA) [ above of=PKA, node distance = \encdist] {$\mySK_A$};
  \node (PKB) [ below of=PKA, node distance = \encdist] {$\myPK_B$};

  \node (enc2) [enc, below of=enc1, node distance=1.5*\ydist] {\myAlgofont{Enc}};
  \node (PKE) [ left of=enc2] {$\myPK_E$};
  \node (SKE) [ above of=PKE, node distance = \encdist] {$\mySK_E$};

  \node (mid) [below of=enc2, node distance=1.5*\ydist, coordinate] {};
  \node (sign) [int, left of=mid] {\myAlgofont{Sign}};
  \node (SKS) [left of=sign] {$\mySK_{B,S}$};

  \node (c)    [below of=mid, node distance=\ydist] {c};

  \draw[->] (m) -- (enc1);

  \draw[->] (SKA) -- ($(enc1.west) + (0,\encdist)$);
  \draw[->] (PKA) -- (enc1);
  \draw[->] (PKB) -- ($(enc1.west) - (0,\encdist)$);

  \draw[->] (SKE) -- ($(enc2.west) + (0,\encdist)$);
  \draw[->] (PKE) -- (enc2);
  \draw[->] ($(PKB.east) + (0.1cm,0)$) |- ($(enc2.west) - (0,\encdist)$);

  \draw[->] (enc1) -- (enc2);
  \draw[->] (enc2) -- node[left, ifelse] {if $\mySK_{B,S}$} (sign);
  \draw[->] (SKS) --  (sign);
  \draw[->] (sign) -- (c);
  \draw[->] (enc2) -- node [right, ifelse] {else}(c);
\end{tikzpicture}

\begin{tikzpicture}
\end{tikzpicture}

  \caption{Sealing of a message $m$ to return a ciphertext $c$.}
  \label{fig:sealing}
\end{figure}

This sophisticated sealing scheme is depicted in \cref{fig:sealing}.
It takes as an input a message $m$, the long term encryption key pair $\myKeyPair{_A}$, the ephemeral keys $\myKeyPair{_E}$, and -- optionally -- the signing key $\mySK_{B,S}$ belonging to the receiver's signing/verification key pair $\myKeyPair{_{B,S}}$.
We first encrypt using the combination of the sender's and the receiver's keys.
We then apply the second layer encryption using ephemeral keys $\myKeyPair{_E}$.
Finally, we check whether a signing key $\mySK_{A,S}$ was passed as input.
If this is the case then we additionally sign the ciphertext before returning it.
Otherwise, we just return the ciphertext without signing it.

Every user shares its verification key $\myPK_{B,S}$ with the server such that the server can check the signatures of incoming messages.
Therefore, the receiver does not have to check it.
The receiver, however, needs to decrypt the outer layer using the private key $\mySK_B$ and the supplied ephemeral key $\myPK_E$.
Then the receiver can extract the sender's public key $\myPK_A$ and use it together with the private key $\mySK_B$ to decrypt the inner layer.

