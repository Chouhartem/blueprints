\subsection{Prevent replay of old patches}

\paragraph{Problem:}
Malicious users with read access can replay old patches drafted by other users.
Since they have a valid signatures, they will be accepted.

\paragraph{Consequences:}
Malicious users can send old checkpoints and thus reset documents.
Malicious users could also generate a lot of traffic which might result in a \ac{DoS}.

\paragraph{Suggestions:}
There are multiple approaches:
\begin{enumerate}
  \item Add a time stamp to patches and accept them only if they are within a certain range to the current time.
    The time stamps must be in the signed part of the message to make them unforgeable.
  \item The server keeps a list of the hash of the last $N$ accepted messages.
    Every incoming message is valid if its hash is not in the list and if it references the hash of one of messages in the list.
  \item Clients that want to send patches have to first prove to the server that they are allowed to do so.
    For that, the server sends them a random value (a \textit{challenge}) which they must sign and return.
    If they succeed, they are marked to have write capabilities and the server accepts their patches.
\end{enumerate}

\paragraph{Drawbacks:}
The drawbacks for the different approaches are the followings:
\begin{enumerate}
  \item Users with a wrong system time won't be able to produce new patches.
  \item
    \begin{itemize}
      \item The mechanism is fairly complex for the server and $N$ has to be chosen carefully, and dependent of the number of users with right access (the more there are, the bigger the chance of simultaneous messages that could result in missing the window).
      \item Some types of data structures built on top of channels (such as mailboxes) contain sequences of messages which are independent of each other.
        For these types of channels it could be inconvenient to have to know the parent patch.
        Replay protection could therefore be enforced selectively with an attribute set in the channel metadata.
    \end{itemize}
  \item The challenge/verification has to be s.t. a \ac{DoS} attack (state exhaustion) against the server is not possible.
\end{enumerate}

