\subsection{Traitor Tracing}

\paragraph{Problem:}
Everyone having read/write access to a document or folder can forward the keys without being noticed by anyone else.

\paragraph{Consequences:}
The access to folders and documents can be delegated without being noticed.
As such team members have partial admin rights since they can share (but not deny) access to documents.

There is currently no way to find who leaked a document.

\paragraph{Suggestions:}
Deploy a scheme inspired by broadcast encryption~\cite{Fiat1993} which generates new decryption and signing keys for every person allowed to access the document.
CryptPad's setting is different from a traditional broadcasting one (i.e., communication over satellites where anybody can listen).
This allows to simplify things, e.g., the server can enforce an access list.
Similarly, there is no central node broadcasting, but potentially many authors interacting with each other.
Among these authors a central user/group is responsible for managing key distribution.

Once a document is leaked, this allows to trace the traitor, i.e., find who leaked the keys, and to block the leaking keys from getting access.
In the same way, per-user signing keys could allow for \enquote{traitor-tracing} and revocation of particular edits in the event of a vandal.
The pure knowledge that this is possible will already discourage leaking keys.

\paragraph{Drawbacks:}
\begin{enumerate}
  \item Traitor tracing contradicts plausible deniability.
  \item One has to be carefully, who should be able to do the traitor tracing. Possible are: owners, everyone with access to a document, or everyone. The last should be avoided in favour of privacy.
  \item Tracing can easily be circumvented by importing/exporting a file or taking screenshots. However, this only leaks \textit{read} access.
\end{enumerate}

